\section{Allgemeine Rahmenbedingungen für die Labornutzung}\label{sec:allgemein}

\noindent
Für alle Labore des Lehrstuhls für Medieninformatik gilt:

\begin{enumerate}
    \item{Personen mit bestätigter Sars-CoV-2-Infektion oder typischen Krankheitssymptomen ist der Zutritt zu allen Räumen des Lehrstuhls für Medieninformatik untersagt.
    Vor jeder Nutzung eines der Labore muss die eigene Symptomatik mit einer Checkliste abgeglichen werden.}
    \item Grundsätzlich ist die Labornutzung nur gestattet, wenn
    \begin{enumerate}
        \item Die Laborinfrastruktur für die Tätigkeit unverzichtbar ist.
        \item Die Tätigkeit zeitlich dringend ist.
        \item Alle Hygienevorschriften eingehalten werden.
    \end{enumerate}
    \item{Jede Labornutzung muss vorher bei den zuständigen Mitarbeitenden des Lehrstuhls für Medieninformatik mit einer ausreichenden Begründung beantragt werden.
    Diese entscheiden einmal pro Woche für jeden Fall individuell anhand eines Kriterienkatalogs, ob die Nutzung des Labors gestattet wird und vergibt Laborzeiten dann die Nutzer*innen.}
    \item{Vor der Erstnutzung eines Labors durch eine Person ist eine persönliche Einweisung in die Arbeits- und Hygienerichtlinien durch die zuständigen Mitarbeitenden des Lehrstuhls für Medieninformatik notwendig.}
    \item{Beginn und Ende jeder Labornutzung sind über ein digitales Formular zu dokumentieren.
    Dabei sind Nutzungszeitraum, Name, Handynummer und E-Mail-Adresse anzugeben.
    Dies gilt auch für Proband*innen von Nutzerstudien.
    Am Ende jeder Benutzung müssen verwendete Gegenstände und Tischoberflächen desinfiziert werden.
    Dies wird in einem Hygieneprotokoll festgehalten.}
    \item{Angemessene Handhygiene, ein Mindestabstand von 1,5 Metern, sowie die Hust- und Niesetikette, sind einzuhalten.
    Ist mehr als eine Person im Raum, ist zusätzlich ein Mund- und Nasenschutz zu tragen.
    Darauf wird in jedem Labor mit Schildern hingewiesen.}
    \item{Fenster sollten wenn möglich offen sein. Räume mit Fenster sind spätestens alle 45 Minuten für mindestens 5 Minuten gründlich zu lüften (Stoßlüften).}
\end{enumerate}
