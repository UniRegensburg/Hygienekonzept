\section{Rahmenbedingungen für die Labornutzung}\label{sec:allgemein}
\emph{\textbf{Änderung: 02.09.2021:} Maskenpflicht gemäß arbeitsschutzrechtlichen Bestimmungen. Verweis auf Lüftungskonzept der Universität Regensburg. Vorgehen zur Überprüfung der 3G-Regel hinzugefügt.}

\noindent
Für alle Labore des Lehrstuhls für Medieninformatik gilt:

\begin{enumerate}
    \item{Personen mit bestätigter Sars-CoV-2-Infektion oder typischen Krankheitssymptomen ist der Zutritt zu allen Räumen des Lehrstuhls für Medieninformatik untersagt. Vor jeder Nutzung eines der Labore muss die eigene Symptomatik mit einer Checkliste abgeglichen werden.}
    \item Grundsätzlich ist die Labornutzung nur gestattet, wenn
    \begin{enumerate}
        \item die Laborinfrastruktur für die Tätigkeit unverzichtbar ist,
        \item die Tätigkeit zeitlich dringend ist und 
        \item alle Hygienevorschriften eingehalten werden.
    \end{enumerate}
    \item{Jede Labornutzung muss vorher bei den zuständigen Mitarbeitenden des Lehrstuhls für Medieninformatik mit einer ausreichenden Begründung beantragt werden. Diese entscheiden einmal pro Woche für jeden Fall individuell anhand eines Kriterienkatalogs, ob die Nutzung des Labors gestattet wird und vergibt Laborzeiten dann an die Nutzer:innen.}
    \item{Vor der Erstnutzung eines Labors durch eine Person ist eine persönliche Einweisung in die Arbeits- und Hygienerichtlinien durch die zuständigen Mitarbeitenden des Lehrstuhls für Medieninformatik notwendig.}
    \item{Beginn\footnote{\url{https://wiki.mi.ur.de/lab/checkin}} und Ende\footnote{\url{https://wiki.mi.ur.de/lab/checkout}} jeder Labornutzung sind über ein digitales Formular zu dokumentieren.
    Dabei sind Nutzungszeitraum, Name, Handynummer und E-Mail-Adresse anzugeben.
    Dies gilt auch für Proband:innen von Nutzerstudien.
    Nach jeder Benutzung müssen verwendete Gegenstände und Tische desinfiziert werden.
    Dies wird in einem Hygieneprotokoll festgehalten.}
    \item{Angemessene Handhygiene, ein Mindestabstand von 1,5 Metern, sowie die Hust- und Niesetikette, sind einzuhalten.

    \sout{Ist mehr als eine Person im Raum, ist zusätzlich ein Mund- und Nasenschutz zu tragen.}

    Die Maskenpflicht gemäß der arbeitsschutzrechtlichen Bestimmungen der Universität (Stand 02.09.2021: FFP2-Maskenpflicht für Personen ab dem 16. Lebensjahr, medizinische Maske für Personen zwischen 6 und 16 Jahren) sind einzuhalten.
    Darauf wird in jedem Labor mit Schildern hingewiesen.}
    \item{Das Lüftungskonzept der Universität wird in allen Laborräumen befolgt.
    Fenster sollten wenn möglich offen sein.
    Räume mit Fenster sind spätestens alle 45 Minuten für mindestens 5 Minuten gründlich zu lüften (Stoßlüften).}
\item{Bei einer 7-Tage-Inzidenz von über 35 ist die 3G-Regelung (geimpft, genesen oder negativ getestet) einzuhalten.
    Dazu muss entweder ein vollständiger Impfschutz beziehungsweise eine vergangene COVID-Infektion nachgewiesen, oder zweimal wöchentlich ein negativer Coronatest vorgelegt werden.

    Die Einhaltung von 3G wird durch die \emph{CovPassCheck-App}\footnote{\url{https://www.digitaler-impfnachweis-app.de/covpasscheck-app/}} des RKI und ein digitales COVID-Zertifikat in Verbindung mit dem Personalausweis nachgewiesen.
    Die Überprüfung von Laborbenutzer:innen wird durch die Laborbetreuer:innen durchgeführt, Proband:innen von Nutzerstudien werden durch jeweilige Versuchsleiter:innen geprüft.

    Aus Datenschutzgründen steht es Laborbenutzer:innen frei, ihren Impf- oder Genesenenstatus in einer Liste dokumentieren zu lassen.
    Stimmen sie dem zu, sind in Zukunft keine weiteren Nachweise nötig; andernfalls muss der 3G-Nachweis vor jeder Laborbenutzung erbracht werden.
    }
\end{enumerate}
