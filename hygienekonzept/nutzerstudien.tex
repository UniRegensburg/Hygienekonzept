\section{Durchführung von Nutzerstudien}\label{sec:nutzerstudien}
\emph{\textbf{Änderung: 01.07.2021:} Quarantänezeit für HMDs und MoCap-Anzüge entfällt. }

\noindent
In der Medieninformatik werden neue Interaktionstechniken und User Interfaces entwickelt und evaluiert, sowie bestehende untersucht und verbessert. Ein etabliertes Vorgehen ist dabei das Durchführen von Nutzerstudien, bei denen Proband:innen in einer kontrollierten Laborumgebung mit diesen Systemen interagieren. Einige Studien werden auch in der virtuellen Realität durchgeführt und setzen spezielle Hardware wie Head-Mounted Displays (HMDs) oder Tracking-Anzüge für ein Motion-Capturing-System voraus.

\medskip
\noindent
Das Durchführen von Nutzerstudien im Labor ist nur dann gestattet, wenn die Laborinfrastruktur unbedingt benötigt wird und die Studie nicht remote durchgeführt werden kann. Vorerst sind Interviews, Fokusgruppen und ähnliche Studien beispielsweise per Videochat durchzuführen.

\noindent
Beim Durchführen von Nutzerstudien sind folgende Maßnahmen zu beachten:

\begin{enumerate}
    \item Name, E-Mail-Adresse und Handynummer aller Proband:innen, sowie der Zeitraum der Studie werden dokumentiert, um mögliche Infektionsketten nachvollziehen zu können. Diese Daten werden nach 30 Tagen gelöscht.
    \item Nach jeder Proband:in werden alle Kontaktgegenstände und -flächen desinfiziert und, wenn möglich, der Laborraum für mindestens fünf Minuten gelüftet.
    \item Auch für Proband:innen gilt: Befindet sich mehr als eine Person im Raum, muss ein Mund- und Nasenschutz getragen werden.
    \item Es wird jederzeit ein Mindestabstand von 1,5 Metern zu Probanden eingehalten.
    \item Nur ein/e Proband:in befindet sich gleichzeitig im Labor. Wenn möglich sollen Proband:innen vor der Studie außerhalb des Gebäudes warten. Um Warteschlangen zu vermeiden, soll ausreichend viel Zeit zwischen einzelnen Proband:innen eingeplant werden.
\end{enumerate}

\noindent
\sout{Nutzerstudien, die das Tragen von Hardware am Körper voraussetzen, weisen ein erhöhtes Risiko der Schmierinfektion auf. Deshalb wird vorerst die Anzahl an ausgeschriebenen Abschluss- und Projektarbeiten, bei denen diese Art von Studien durchgeführt werden, reduziert. Alternative Themen, Projektideen und Durchführungsmethoden werden angeboten.
Für Arbeiten, für die keine alternativen Methoden existieren und für die das Benutzen von am Körper getragener Hardware erforderlich ist, dürfen Nutzerstudien lediglich unter erhöhten Sicherheitsmaßnahmen durchgeführt werden.}

Vorerst wird die Anzahl an ausgeschriebenen Abschluss- und Projektarbeiten, bei denen Präsenz im Labor nötig ist, reduziert. Alternative Themen, Projektideen und Durchführungsmethoden werden angeboten.
Für Arbeiten, für die keine alternativen Methoden existieren und für die das Benutzen von am Körper getragener Hardware erforderlich ist, dürfen Nutzerstudien lediglich unter erhöhten Sicherheitsmaßnahmen durchgeführt werden.

\subsection{Durchführung von Studien mit Head-Mounted Displays}\label{subsec:nutzerstudien_hmd}

Da durch die Verwendung von Head-Mounted Displays (HMD) ein höheres Infektionsrisiko besteht, ist bei Nutzerstudien mit derartigen Geräten ein genaues Ablaufprotokoll einzuhalten.

\subsubsection*{Vorbereiten und Anbringen des HMD}

{
\singlespacing
\begin{itemize}
    \item \sout{Sauberes HMD (nach 72-stündiger Quarantäne) aus verschließbarer Box herausholen}
    \item HMD mit einem “Einmalüberzug” (wird nach einmaliger Benutzung entsorgt) vorbereiten
    \item Proband:in auf markierten Platz verweisen 
    \item Versuchsleiter:in trägt \sout{Handschuhe und} Mund-Nasen-Schutz (MNS)
    \item Versuchsleiter:in tritt ausschließlich seitlich an Proband:in heran
    \item Versuchsleiter:in hält HMD von unten am Gehäuse und bittet Proband:in sich nach vorne mit Gesicht auf die Brille zu beugen 
    \item Versuchsleiter:in befestigt Kopfgurt 
    \item Controller liegt vor Proband:in bereit 
    \item \sout{Versuchsleiter:in zieht Handschuhe aus und entsorgt diese in verschließbaren Mülleimer}
    \item Versuchsleiter:in desinfiziert sich die Hände
    \item Versuchsleiter:in setzt sich an den Kontrollrechner in mindestens 1,5 Meter Abstand zu Proband:in
\end{itemize}
}

\subsubsection*{Abnehmen des HMD}

{
\singlespacing
\begin{itemize}
    \item Versuchsleiter:in trägt \sout{frische Handschuhe (und MNS)} Mund-Nasen-Schutz
    \item Versuchsleiter:in tritt seitlich an Proband:in
    \item Versuchsleiter:in lockert Kopfgurt
    \item Proband:in zieht dann Brille nach vorne weg
    \item HMD Brillensheet und Handschuhe in verschließbaren Mülleimer entsorgen
    \item \sout{HMD für 72 Stunden in Quarantäne setzen}
    \begin{itemize}
        \item \sout{Verschließbare Box öffnen und HMD hineinlegen}
        \item \sout{Box mit Datum und Uhrzeit versehen}
    \end{itemize}
    \item Versuchsleiter:in und Proband:in desinfizieren sich die Hände
\end{itemize}
}

\subsection{Durchführung von Studien mit Tracking-Anzügen}\label{subsec:nutzerstudien_mocap}


Wird bei einer Studie das Motion-Capturing-System verwendet, so müssen Proband:innen Ganzkörperanzüge mit speziellen Markern tragen.
Durch den direkten Körperkontakt besteht ein erhöhtes Infektionsrisiko, weshalb ein genaues Ablaufprotokoll zum An- und Ablegen des Anzugs sowie zu dessen Reinigung und Aufbewahrung einzuhalten ist.

\subsubsection*{Zu Beginn der Studie}

{
\singlespacing
\begin{itemize}
    \item Versuchsleiter:in trägt \sout{Handschuhe (und MNS)} Mund-Nasen-Schutz
    \item Anzug für Proband:in vorbereiten und zur Verfügung stellen
    \item Proband:in auf markierten Platz verweisen 
    \item Versuchsleiter:in beschreibt Schritt für Schritt wie man den Motion Capture Anzug anzieht
\end{itemize}
}

\subsubsection*{Nach der Studie}

{
\singlespacing
\begin{itemize}
    \item \sout{Motion Capture Anzüge in Quarantäne setzen}
    \item \sout{verschließbare Box für Anzüge öffnen und Anzug hineinlegen}
    \item \sout{Anzug in Box zur Reinigung bringen}
\end{itemize}
}

Gebrauchte Anzüge werden regelmäßig vom Laborpersonal gereinigt.

