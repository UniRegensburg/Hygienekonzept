\section{Arbeit mit Head-Mounted Displays (HMD)}\label{sec:hmd}
\emph{\textbf{Änderung: 02.07.2021:} Quarantänezeit für HMDs entfällt.}

\medskip
\noindent
Anwendungen im Bereich der Virtual und Augmented Reality sind Teil vieler Arbeiten in der Medieninformatik.
Bei der Entwicklung von Anwendungen für HMDs wird durch personalisierte Geräte gewährleistet, dass ein HMD nur von ein und derselben Person verwendet wird.
Nach der Verwendung von HMDs, beispielsweise für die Entwicklung von Anwendungen oder für Proband:innen bei Nutzerstudien, werden diese desinfiziert.

\sout{Während mit dem Gerät nicht gearbeitet wird, wird es zunächst desinfiziert und dann einzeln in einer beschrifteten und verschließbaren Box aufbewahrt. Bevor eine andere Person dieses Gerät benutzen darf, muss es sich für mindestens 72 Stunden in dieser Box befinden, damit mögliche Keime absterben.}

\medskip
\noindent
Bei der Durchführung von Nutzerstudien, bei denen Proband:innen ein HMD tragen, ist dem in Kapitel \ref{subsec:nutzerstudien_hmd} beschriebenen Ablauf zu folgen.
