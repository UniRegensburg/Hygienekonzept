\clearpage

\appendix

\section{Verantwortliche und Ansprechpartner}\label{app:verantwortliche}

\noindent
Prof. Dr. Christian Wolff, der Inhaber des Lehrstuhls für Medieninformatik, überträgt die Verantwortung für die Umsetzung des Hygienekonzepts an ein Gremium von Mitarbeitenden des Lehrstuhls, namentlich

\begin{itemize}
    \item Prof. Dr. Christian Wolff (christian.wolff@ur.de)
    \item Prof. Dr. Niels Henze (niels.henze@ur.de)
    \item Dr. Raphael Wimmer (raphael.wimmer@ur.de)
    \item Alexander Bazo (alexander.bazo@ur.de)
    \item Patricia Böhm (patricia.boehm@ur.de)
    \item Victoria Böhm (victoria.boehm@ur.de)
    \item Martin Brockelmann (martin.brockelmann@ur.de)
    \item Martin Kocur (martin.kocur@ur.de)
    \item Andreas Schmid (andreas.schmid@ur.de)
    \item Thomas Schmidt (thomas.schmidt@ur.de)
\end{itemize}

\noindent
Zudem dienen studentische Hilfskräfte als Ansprechpartner für Studierende.
Sie unterstützen die Mitarbeitenden des Lehrstuhls außerdem bei der praktischen Umsetzung der Hygienemaßnahmen.

Diese Hilfskräfte sind:

\begin{itemize}
    \item Christoph Härtl (Future Interaction Lab, christoph.haertl@stud.uni-regensburg.de)
    \item Manuel Mayer (VR4 Studio, manuel.mayer@stud.uni-regensburg.de)
\end{itemize}

\section{Hinweisschilder und Aushänge}\label{app:hinweisschilder}

\subsection{Hinweisschild: Hygienehinweise}\label{app:hygiene}

\frame{\includegraphics[width=\textwidth]{{hygienehinweis}}}

\subsection{Hinweisschild: Symptome}\label{app:symptome}

\frame{\includegraphics[width=\textwidth]{{symptome}}}

\subsection{Hinweisschild: VR-Nutzerstudien}\label{app:vr}

\frame{\includegraphics[width=\textwidth]{{hmd}}}

\subsection{Hinweisschild: Check-in}\label{app:checkin}

\frame{\includegraphics[width=\textwidth]{{checkin}}}

\subsection{Hinweisschild: Check-in (Proband:innen)}\label{app:checkin_gast}

\frame{\includegraphics[width=\textwidth]{{checkin_gast}}}

\subsection{Hinweisschild: Check-out}\label{app:checkout}

\frame{\includegraphics[width=\textwidth]{{checkout}}}

\section{Raumpläne}\label{app:raumpläne}

\begin{center}
    \sffamily
    \frame{
    \begin{tabular}{ m{1.1cm} m{3cm} }
        \multicolumn{2}{l}{\textbf{Legende}} \\
        \vspace{2mm}
        \includegraphics[width=1cm]{arbeitsplatz} & Arbeitsplatz \\
        \includegraphics[width=1cm]{tisch} & Tisch \\
        \includegraphics[width=1cm]{durchgang} & Tür \\
        \includegraphics[width=1cm]{fenster} & Fenster \\
        \includegraphics[width=1cm]{trennwand} & Trennwand \\
    \end{tabular}
    \quad
    \begin{tabular}{ m{1.1cm} m{5cm} }
        \vspace{2mm}
        \includegraphics[width=1cm]{flächendesinfektion} & Flächendesinfektionsmittel \\
        \includegraphics[width=1cm]{handdesinfektion} & Händedesinfektionsmittel \\
        \includegraphics[width=1cm]{waschbecken} & Waschbecken \\
        \includegraphics[width=1cm]{hinweisschilder} & Hinweisschilder \\
    \end{tabular}
    }
\end{center}

\subsection{Future Interaction Lab}\label{app:plan_fil}

\frame{\includegraphics[width=\textwidth]{{fil_konzept}}}

\subsection{PT 3.0.30}\label{app:plan_pt3.0.30}

\frame{\includegraphics[width=\textwidth]{{pt_3_0_30_konzept}}}

\subsection{Eyetracking-Classroom}\label{app:plan_eyetracking}

\frame{\includegraphics[width=\textwidth]{{eyetracking-classroom_konzept}}}

\subsection{TechBase: VR4}\label{app:plan_vr4}

\frame{\includegraphics[width=\textwidth]{{vr4_konzept}}}

