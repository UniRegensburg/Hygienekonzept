\section{Maßnahmen zur praktischen Umsetzung der Regeln}

\begin{enumerate}
    \item Die Laborräume werden nur an Einzelpersonen vergeben, Arbeit in Projektgruppen ist nicht gestattet.
    \item Wenn möglich werden Räume für längere zeitliche Blöcke am Stück vergeben, sodass möglichst wenig personeller Wechsel in den Laboren stattfindet.
    \item Hardware und Werkzeug werden wenn möglich personalisiert vergeben, sodass sie während eines Zeitraums nur von einer Person verwendet werden.
    \item In allen Laborräumen werden Händedesinfektionsmittelspender, Sprühflaschen mit Flächendesinfektionsmittel und Einwegmasken bereitgestellt.
    \item Arbeitsbereiche in den Laboren werden so eingerichtet, dass ein Mindestabstand von 1,5 Metern besteht. Zusätzliche Tische und Stühle werden wenn möglich entfernt, sodass die Räume eine geringe Affordanz bieten, gegen den Mindestabstand und die maximale Personenzahl zu verstoßen.
    \item In jedem Labor werden folgende Hinweisschilder angebracht:
    \begin{enumerate}
        \item Handhygiene
        \item Hust- und Niesetikette
        \item Abstandsregeln
        \item Maskenpflicht
        \item Kurzfassung Hygieneregeln
        \item Symptome COVID-19
    \end{enumerate}
    \item Das Einhalten der Regeln wird vom Laborpersonal regelmäßig kontrolliert.
    \item Wir behalten uns vor, Verstöße gegen die Hygieneregeln mit Ausschluss aus dem Laborbetrieb zu ahnden.
\end{enumerate}
