\documentclass[hidelinks,12pt]{extarticle}

\usepackage[a4paper,left=3cm,right=3cm,top=2cm,bottom=2.5cm]{geometry}
\usepackage{graphicx}
\usepackage{titlesec}
\usepackage{authblk}
\usepackage[utf8]{inputenc}
\usepackage[T1]{fontenc}
\usepackage[ngerman]{babel}
\usepackage{setspace}
\usepackage{unicode-math}
\usepackage{fontspec}
\usepackage{tabularx}
\usepackage{multirow}
\usepackage{hyperref}

\setmainfont{TeX Gyre Pagella}
\setsansfont{TeX Gyre Heros}
\setmonofont{TeX Gyre Cursor}

\graphicspath{{../supplements/img/}, {../supplements/}, {../raumpläne/}}

\renewcommand\Authand{ und }
\renewcommand\Authands{ und }

\newcommand{\labinfo}[3]{
    \begin{tabularx}{\textwidth}{lX}
        \textbf{Art der Nutzung} & #1 \\
        \textbf{Arbeitsplätze} & #2 \\
        \textbf{Verantwortlich} & #3 \\
    \end{tabularx}

    \vspace{3mm}
}

\title{
    \bfseries
    \sffamily
    Hygienekonzept für die Labore des Lehrstuhls für Medieninformatik
}

%\author{Christian Wolff \\ \href{mailto:christian.wolff@ur.de}{christian.wolff@ur.de}}
\author{Andreas Schmid}
\author{Alexander Bazo}
\author{Martin Kocur}
\author{Raphael Wimmer}
\author{Martin Brockelmann}
\author{Patricia Böhm}
\author{Thomas Schmidt}
\author{Victoria Böhm}
\author{Christian Wolff}

\affil{Lehrstuhl für Medieninformatik, Universität Regensburg}

\begin{document}

\onehalfspacing
\maketitle

\noindent
Um die Labore des Lehrstuhls für Medieninformatik zu benutzen, sind die in diesem Dokument aufgeführten Hygieneregeln einzuhalten.
Werden die Labore für Nutzerstudien verwendet, so gelten zusätzliche Vorsichtsmaßnahmen.
Die Hygieneregeln wurden passend zur COVID-19-Situation am 2. Juni 2020 aufgestellt.
Sie werden regelmäßig überprüft und an die aktuelle Situation angepasst.

\newpage
\tableofcontents 

\section{Allgemeine Informationen zu diesem Dokument}

Zu den Laboren des Lehrstuhls für Medieninformatik zählen:

\begin{itemize}
    \item Das Future Interaction Lab
    \begin{itemize}
        \item Usability-Labor (PT 3.0.27)
        \item Werkstatt (PT 3.0.28)
        \item Besprechungsraum (PT 3.0.28A)
    \end{itemize}
    \item Der Eyetracking-Classroom (SG 5.20/21)
    \item Versuchsraum 4 in der TechBase
    \begin{itemize}
        \item VR4 Labor
        \item VR4 Studio
        \item VR4 Werkstatt
    \end{itemize}
\end{itemize}

\noindent
Dieses Dokument enthält

\begin{itemize}
    \item Allgemeine Rahmenbedingungen und Hygieneregeln zur Nutzung der Labore
    \item Maßnahmen des Lehrstuhls, um die Umsetzung der Hygienemaßnahmen in den Laboren zu gewährleisten
    \item Spezifische Hinweise zum Benutzen von Head-Mounted Displays (HMD)
    \item Zusätzliche Regeln für die Durchführung von Nutzerstudien
    \item Individuelle Regelungen für jedes der Labore
\end{itemize}

\noindent
Zusätzlich hängen diesem Dokument an

\begin{itemize}
    \item Raumpläne der einzelnen Labore
    \item Hinweisschilder zu den Hygienemaßnahmen
\end{itemize}


\section{Allgemeine Rahmenbedingungen für die Labornutzung}\label{sec:allgemein}

\noindent
Für alle Labore des Lehrstuhls für Medieninformatik gilt:

\begin{enumerate}
    \item{Personen mit bestätigter Sars-CoV-2-Infektion oder typischen Krankheitssymptomen ist der Zutritt zu allen Räumen des Lehrstuhls für Medieninformatik untersagt.
    Vor jeder Nutzung eines der Labore muss die eigene Symptomatik mit einer Checkliste abgeglichen werden.}
    \item Grundsätzlich ist die Labornutzung nur gestattet, wenn
    \begin{enumerate}
        \item Die Laborinfrastruktur für die Tätigkeit unverzichtbar ist.
        \item Die Tätigkeit zeitlich dringend ist.
        \item Alle Hygienevorschriften eingehalten werden.
    \end{enumerate}
    \item{Jede Labornutzung muss vorher bei den zuständigen Mitarbeitenden des Lehrstuhls für Medieninformatik mit einer ausreichenden Begründung beantragt werden.
    Diese entscheiden einmal pro Woche für jeden Fall individuell anhand eines Kriterienkatalogs, ob die Nutzung des Labors gestattet wird und vergibt Laborzeiten dann die Nutzer*innen.}
    \item{Vor der Erstnutzung eines Labors durch eine Person ist eine persönliche Einweisung in die Arbeits- und Hygienerichtlinien durch die zuständigen Mitarbeitenden des Lehrstuhls für Medieninformatik notwendig.}
    \item{Beginn und Ende jeder Labornutzung sind über ein digitales Formular zu dokumentieren.
    Dabei sind Nutzungszeitraum, Name, Handynummer und E-Mail-Adresse anzugeben.
    Dies gilt auch für Proband*innen von Nutzerstudien.
    Am Ende jeder Benutzung müssen verwendete Gegenstände und Tischoberflächen desinfiziert werden.
    Dies wird in einem Hygieneprotokoll festgehalten.}
    \item{Angemessene Handhygiene, ein Mindestabstand von 1,5 Metern, sowie die Hust- und Niesetikette, sind einzuhalten.
    Ist mehr als eine Person im Raum, ist zusätzlich ein Mund- und Nasenschutz zu tragen.
    Darauf wird in jedem Labor mit Schildern hingewiesen.}
    \item{Fenster sollten wenn möglich offen sein. Räume mit Fenster sind spätestens alle 45 Minuten für mindestens 5 Minuten gründlich zu lüften (Stoßlüften).}
\end{enumerate}


\section{Maßnahmen zur praktischen Umsetzung der Regeln}\label{sec:umsetzung}

\begin{enumerate}
    \item Die Laborräume werden nur an Einzelpersonen vergeben, Arbeit in Projektgruppen ist nicht gestattet.
    \item Wenn möglich, werden Räume für längere zeitliche Blöcke am Stück vergeben, sodass möglichst wenig personeller Wechsel in den Laboren stattfindet.
    \item Hardware und Werkzeug werden, wenn möglich, personalisiert vergeben, sodass sie während eines Zeitraums nur von einer Person verwendet werden.
    \item In allen Laborräumen werden Händedesinfektionsmittelspender, Sprühflaschen mit Flächendesinfektionsmittel, Einweghandschuhe und Einwegmasken bereitgestellt. 
    \item Arbeitsbereiche in den Laboren werden so eingerichtet, dass ein Mindestabstand von 1,5 Metern besteht. Zusätzliche Tische und Stühle werden, wenn möglich, entfernt, sodass die Räume die Einhaltung von  Mindestabstand und maximaler Personenzahl erleichtern.
    \item In jedem Labor werden folgende Hinweisschilder angebracht:
    \begin{enumerate}
        \item Handhygiene
        \item Hust- und Niesetikette
        \item Abstandsregeln
        \item Maskenpflicht
        \item Kurzfassung Hygieneregeln
        \item Symptome COVID-19
    \end{enumerate}
    \item Das Einhalten der Regeln wird vom Laborpersonal regelmäßig kontrolliert.
    \item Wir behalten uns vor, Verstöße gegen die Hygieneregeln mit Ausschluss aus dem Laborbetrieb zu ahnden.
\end{enumerate}


\section{Arbeit mit Head-Mounted Displays (HMD)}\label{sec:hmd}

\medskip
\noindent
Anwendungen im Bereich der Virtual und Augmented Reality sind Teil vieler Arbeiten in der Medieninformatik.
Bei der Entwicklung von Anwendungen für HMDs wird durch personalisierte Geräte gewährleistet, dass ein HMD nur von ein und derselben Person verwendet wird.
Nach der Verwendung von HMDs, beispielsweise für die Entwicklung von Anwendungen oder für Proband:innen bei Nutzerstudien, werden diese desinfiziert.

\medskip
\noindent
Bei der Durchführung von Nutzerstudien, bei denen Proband:innen ein HMD tragen, ist dem in Kapitel \ref{subsec:nutzerstudien_hmd} beschriebenen Ablauf zu folgen.


\section{Durchführung von Nutzerstudien}\label{sec:nutzerstudien}
\emph{\textbf{Änderung: 01.07.2021:} Quarantänezeit für HMDs und MoCap-Anzüge entfällt. }

\noindent
In der Medieninformatik werden neue Interaktionstechniken und User Interfaces entwickelt und evaluiert, sowie bestehende untersucht und verbessert. Ein etabliertes Vorgehen ist dabei das Durchführen von Nutzerstudien, bei denen Proband:innen in einer kontrollierten Laborumgebung mit diesen Systemen interagieren. Einige Studien werden auch in der virtuellen Realität durchgeführt und setzen spezielle Hardware wie Head-Mounted Displays (HMDs) oder Tracking-Anzüge für ein Motion-Capturing-System voraus.

\medskip
\noindent
Das Durchführen von Nutzerstudien im Labor ist nur dann gestattet, wenn die Laborinfrastruktur unbedingt benötigt wird und die Studie nicht remote durchgeführt werden kann. Vorerst sind Interviews, Fokusgruppen und ähnliche Studien beispielsweise per Videochat durchzuführen.

\noindent
Beim Durchführen von Nutzerstudien sind folgende Maßnahmen zu beachten:

\begin{enumerate}
    \item Name, E-Mail-Adresse und Handynummer aller Proband:innen, sowie der Zeitraum der Studie werden dokumentiert, um mögliche Infektionsketten nachvollziehen zu können. Diese Daten werden nach 30 Tagen gelöscht.
    \item Nach jeder Proband:in werden alle Kontaktgegenstände und -flächen desinfiziert und, wenn möglich, der Laborraum für mindestens fünf Minuten gelüftet.
    \item Auch für Proband:innen gilt: Befindet sich mehr als eine Person im Raum, muss ein Mund- und Nasenschutz getragen werden.
    \item Es wird jederzeit ein Mindestabstand von 1,5 Metern zu Probanden eingehalten.
    \item Nur ein/e Proband:in befindet sich gleichzeitig im Labor. Wenn möglich sollen Proband:innen vor der Studie außerhalb des Gebäudes warten. Um Warteschlangen zu vermeiden, soll ausreichend viel Zeit zwischen einzelnen Proband:innen eingeplant werden.
\end{enumerate}

\noindent
\sout{Nutzerstudien, die das Tragen von Hardware am Körper voraussetzen, weisen ein erhöhtes Risiko der Schmierinfektion auf. Deshalb wird vorerst die Anzahl an ausgeschriebenen Abschluss- und Projektarbeiten, bei denen diese Art von Studien durchgeführt werden, reduziert. Alternative Themen, Projektideen und Durchführungsmethoden werden angeboten.
Für Arbeiten, für die keine alternativen Methoden existieren und für die das Benutzen von am Körper getragener Hardware erforderlich ist, dürfen Nutzerstudien lediglich unter erhöhten Sicherheitsmaßnahmen durchgeführt werden.}

Vorerst wird die Anzahl an ausgeschriebenen Abschluss- und Projektarbeiten, bei denen Präsenz im Labor nötig ist, reduziert. Alternative Themen, Projektideen und Durchführungsmethoden werden angeboten.
Für Arbeiten, für die keine alternativen Methoden existieren und für die das Benutzen von am Körper getragener Hardware erforderlich ist, dürfen Nutzerstudien lediglich unter erhöhten Sicherheitsmaßnahmen durchgeführt werden.

\subsection{Durchführung von Studien mit Head-Mounted Displays}\label{subsec:nutzerstudien_hmd}

Da durch die Verwendung von Head-Mounted Displays (HMD) ein höheres Infektionsrisiko besteht, ist bei Nutzerstudien mit derartigen Geräten ein genaues Ablaufprotokoll einzuhalten.

\subsubsection*{Vorbereiten und Anbringen des HMD}

{
\singlespacing
\begin{itemize}
    \item \sout{Sauberes HMD (nach 72-stündiger Quarantäne) aus verschließbarer Box herausholen}
    \item HMD mit einem “Einmalüberzug” (wird nach einmaliger Benutzung entsorgt) vorbereiten
    \item Proband:in auf markierten Platz verweisen 
    \item Versuchsleiter:in trägt \sout{Handschuhe und} Mund-Nasen-Schutz (MNS)
    \item Versuchsleiter:in tritt ausschließlich seitlich an Proband:in heran
    \item Versuchsleiter:in hält HMD von unten am Gehäuse und bittet Proband:in sich nach vorne mit Gesicht auf die Brille zu beugen 
    \item Versuchsleiter:in befestigt Kopfgurt 
    \item Controller liegt vor Proband:in bereit 
    \item \sout{Versuchsleiter:in zieht Handschuhe aus und entsorgt diese in verschließbaren Mülleimer}
    \item Versuchsleiter:in desinfiziert sich die Hände
    \item Versuchsleiter:in setzt sich an den Kontrollrechner in mindestens 1,5 Meter Abstand zu Proband:in
\end{itemize}
}

\subsubsection*{Abnehmen des HMD}

{
\singlespacing
\begin{itemize}
    \item Versuchsleiter:in trägt \sout{frische Handschuhe (und MNS)} Mund-Nasen-Schutz
    \item Versuchsleiter:in tritt seitlich an Proband:in
    \item Versuchsleiter:in lockert Kopfgurt
    \item Proband:in zieht dann Brille nach vorne weg
    \item HMD Brillensheet und Handschuhe in verschließbaren Mülleimer entsorgen
    \item \sout{HMD für 72 Stunden in Quarantäne setzen}
    \begin{itemize}
        \item \sout{Verschließbare Box öffnen und HMD hineinlegen}
        \item \sout{Box mit Datum und Uhrzeit versehen}
    \end{itemize}
    \item Versuchsleiter:in und Proband:in desinfizieren sich die Hände
\end{itemize}
}

\subsection{Durchführung von Studien mit Tracking-Anzügen}\label{subsec:nutzerstudien_mocap}


Wird bei einer Studie das Motion-Capturing-System verwendet, so müssen Proband:innen Ganzkörperanzüge mit speziellen Markern tragen.
Durch den direkten Körperkontakt besteht ein erhöhtes Infektionsrisiko, weshalb ein genaues Ablaufprotokoll zum An- und Ablegen des Anzugs sowie zu dessen Reinigung und Aufbewahrung einzuhalten ist.

\subsubsection*{Zu Beginn der Studie}

{
\singlespacing
\begin{itemize}
    \item Versuchsleiter:in trägt \sout{Handschuhe (und MNS)} Mund-Nasen-Schutz
    \item Anzug für Proband:in vorbereiten und zur Verfügung stellen
    \item Proband:in auf markierten Platz verweisen 
    \item Versuchsleiter:in beschreibt Schritt für Schritt wie man den Motion Capture Anzug anzieht
\end{itemize}
}

\subsubsection*{Nach der Studie}

{
\singlespacing
\begin{itemize}
    \item \sout{Motion Capture Anzüge in Quarantäne setzen}
    \item \sout{verschließbare Box für Anzüge öffnen und Anzug hineinlegen}
    \item \sout{Anzug in Box zur Reinigung bringen}
\end{itemize}
}

Gebrauchte Anzüge werden regelmäßig vom Laborpersonal gereinigt.



\section{Individuelle Richtlinien für jedes der Labore}

\subsection{Future Interaction Lab: Usability-Labor}

\labinfo{Klassische Benutzerstudie, Entwicklungsarbeiten, Medienproduktion}{1}{Alexander Bazo, Christoph Härtl}

\noindent
Das Usability-Labor dient vorrangig der Durchführung von Nutzerstudien, insbesondere solcher Experimente, die eine “Alltagssituation” als Testumgebung erfordern.
Dazu sind neben einem klassischen Arbeitsplatz auch ein Sofa und Fernseher vorhanden.
Der Arbeitsplatz kann zusätzlich auch für Entwicklungsarbeiten und die Medienproduktion eingesetzt werden.
Bei Verwendung der vorhandenen Geräte (Workstations) muss nach Gebrauch eine Desinfektion der Eingabegeräte (Maus \& Tastatur) erfolgen.

\medskip
\noindent
Bei Verwendung des Raums für die Durchführung von Nutzerstudien wird die Fläche  durch Öffnen der Trennwand zu der angrenzenden Werkstatt erweitert.
Damit ist dann auch eine Trennung von Ein- und Ausgängen für die Testpersonen möglich.
Die räumlichen und einrichtungstechnischen  Beschränkungen im Labor schließen Studien mit mehr als einem Probanden aus.
Insgesamt sollten im Raum nicht mehr als zwei Personen (ProbandIn und TestleiterIn) anwesend sein.
Denkbar sind zusätzliche BeobachterIn in einem der angrenzenden Räume.
Entsprechende Videotechnik steht im Labor zur Verfügung.

\subsection{Future Interaction Lab: Werkstatt}

\labinfo{Entwicklungsarbeiten, Medienproduktion}{2}{Alexander Bazo, Christoph Härtl}

\noindent
Die Werkstatt im Future Interaction Lab dient vorrangig der (Software-) Entwicklungsarbeit.
Zusätzlich befindet sich hier eine ca. 2,5 x 2,5m große Freifläche, die für Tests- und Studien von VR-Arbeiten verwendet werden kann.
Die Arbeitsplätze sind an einer der Längsseiten angebracht.
Unter Berücksichtigung der notwendigen Abstandsregeln können hier bis zu zwei Personen gleichzeitig arbeiten.
Die Arbeitsflächen wurden weitestgehend freigeräumt, um eine einfache Reinigung und Desinfektion zu erlauben.
NutzerInnen werden angehalten, diesen Zustand beizubehalten.
Bei Verwendung der vorhandenen Geräte (Workstations) muss nach Gebrauch eine Desinfektion der Eingabegeräte (Maus \& Tastatur) erfolgen.
Hier kann zusätzliche Hardware für den regelmäßigen Austausch der angeschlossenen Geräte verwendet werden.
Grundsätzlich kann auch eine Nutzung der Workstations nur bei Verwendung eigener Eingabegeräte gestattet werden.

\medskip
\noindent
Bei Verwendung des Raums für die Durchführung von Nutzerstudien wird die Fläche  durch Öffnen der Trennwand zu einem der angrenzenden Räume (Besprechungsraum oder Usability-Raum) erweitert.
Damit ist dann auch eine Trennung von Ein- und Ausgängen für die Testpersonen möglich.

\subsection{Future Interaction Lab: Besprechungsraum}

\labinfo{Interviews, Projektbesprechung}{Konferenztisch mit 5 Sitzplätzen}{Alexander Bazo, Christoph Härtl}

TODO

\subsection{Eyetracking-Classroom}

\labinfo{Eye-Tracking-Experimente, auch mit gleichzeitiger Nutzung durch mehrere NutzerInnen}{4 (Zusätzliche Arbeitsplätze für Entwicklungsarbeit im Nebenraum vorhanden)}{Alexander Bazo (UR), Forian Hauser (OTH)}

\noindent
Im Classroom stehen 11 Hochleistungs-Eyetracker an separaten Arbeitsplätzen bereit.
Diese sind in Form eines klassischen CIP-Pool-Layouts angeordnet (Vier Sitzreihen mit je 2 bis 3 Arbeitsplätzen).
Bei entsprechender Einhaltung der Abstandsregel beim Eintritt in den Raum kann pro Sitzreihe ein Arbeitsplatz genutzt werden.
Im Idealfall können so auch Experimente durchgeführt werden, die eine gleichzeitige Nutzung des Labors durch mehrere Probanden erfordern.
Durch den angeschlossenen Nebenraum können separate Ein- und Ausgänge für den Laborraum geschaffen werden.
Nach der Verwendung der Arbeitsplätze werden diese desinfiziert, insbesondere die Tastaturen, Monitore und Oberflächen der Eye-Tracker.
Durch die Reduzierung der verwendeten Eye-Tracker pro Sitzreihe können die verwendeten Geräte zwischen einzelnen Studien alterniert werden.

\medskip

\noindent
\textbf{Jedwede Verwendung des Classrooms wird zwischen den Betreibern (OTH Regensburg, Prof. Mottok und Uni Regensburg, Prof. Wolff und Prof. Gruber) abgestimmt.}

\subsection{TechBase VR4: Labor}

\labinfo{Entwicklung von Prototypen, Arbeit mit interaktiven Tischen}{2}{Andreas Schmid, Raphael Wimmer}

\noindent
Das Labor ist der Hauptraum des VR4.
Er bietet genug Platz damit bis zu zwei Personen gleichzeitig darin arbeiten können, ohne den Mindestabstand zu unterschreiten.
In diesem Raum wird vorwiegend im Rahmen von Abschlussarbeiten an Projekten mit interaktiven Tischen und Projektionen gearbeitet, die auf die bestehende Laborinfrastruktur (beispielsweise Traversen) angewiesen sind.
Es werden persönliche Arbeitsbereiche für jede/n Labornutzer*In eingerichtet, welche nur von diesen benutzt werden.
Diese Arbeitsbereiche werden so platziert, dass noch genügend Abstand zum Durchgang Richtung Studio und Werkstatt besteht.

\subsection{TechBase VR4: Studio}

\labinfo{Entwicklung von VR-Anwendungen, Motion Capturing, Nutzerstudien}{2}{Martin Kocur, Polina Ugnivenko, Sarah Graf}
\subsection{TechBase VR4: Werkstatt}

\labinfo{Bau von Prototypen, Elektronikarbeitsplätze}{1}{Andreas Schmid, Raphael Wimmer}

\noindent
Die Werkstatt wird genutzt um Prototypen für neuartige Geräte zu bauen.
Dazu stehen Werkzeuge wie Lötstationen, Netzteile, ein Oszilloskop, Sägen und Akkuschrauber zur Verfügung.
Grundsätzlich ist die Nutzung der Werkstatt gestattet, jedoch kann aufgrund der geringen Größe nur eine Person gleichzeitig darin arbeiten.
Da Werkzeug und Bauteile teilweise schwer zu desinfizieren sind, wird auf große zeitliche Abstände bei der konsekutiven Nutzung geachtet um das Risiko von Schmierinfektionen zu verringern.
Die Tür vom Labor zur Werkstatt darf nur zum Betreten und Verlassen des Raums geöffnet werden.


\clearpage

\appendix

\section{Verantwortliche und Ansprechpartner}\label{app:verantwortliche}

\subsection*{Future Interaction Lab}

\begin{tabularx}{1.1\textwidth}{lrX}
    \textbf{Verantwortlich} & Alexander Bazo & alexander.bazo@ur.de \\
    \textbf{Ansprechpartner} & Christoph Härtl & christoph.haertl@stud.uni-regensburg.de \\
\end{tabularx}

\subsection*{Eyetracking-Classroom}

\begin{tabularx}{\textwidth}{lrX}
    \textbf{Verantwortlich} & Alexander Bazo & alexander.bazo@ur.de \\
\end{tabularx}

\subsection*{TechBase VR4: Labor und Werkstatt}

\begin{tabularx}{\textwidth}{lrX}
    \textbf{Verantwortlich} & Raphael Wimmer & raphael.wimmer@ur.de \\
    & Andreas Schmid & andreas.schmid@ur.de \\
\end{tabularx}

\subsection*{TechBase VR4: Studio}

\begin{tabularx}{1.1\textwidth}{lrX}
    \textbf{Verantwortlich} & Martin Kocur & martin.kocur@ur.de \\
    & Martin Brockelmann & martin.brockelmann@ur.de \\
    \textbf{Ansprechpartner} & Sarah Graf & sarah.graf@stud.uni-regensburg.de \\
    & Polina Ugnivenko & polina.ugnivenko@stud.uni-regensburg.de \\
\end{tabularx}

\section{Hinweisschilder und Aushänge}\label{app:hinweisschilder}

\subsection{Hinweisschild: Hygienehinweise}\label{app:hygiene}

\frame{\includegraphics[width=\textwidth]{{hygienehinweis}}}

\subsection{Hinweisschild: Symptome}\label{app:symptome}

\frame{\includegraphics[width=\textwidth]{{symptome}}}

\subsection{Hinweisschild: VR-Nutzerstudien}\label{app:vr}

\frame{\includegraphics[width=\textwidth]{{hmd}}}

\subsection{Hinweisschild: Check-in}\label{app:checkin}

\frame{\includegraphics[width=\textwidth]{{checkin}}}

\subsection{Hinweisschild: Check-out}\label{app:checkout}

\frame{\includegraphics[width=\textwidth]{{checkout}}}

\section{Raumpläne}\label{app:raumpläne}

\begin{center}
    \sffamily
    \frame{
    \begin{tabular}{ m{1.1cm} m{3cm} }
        \multicolumn{2}{l}{\textbf{Legende}} \\
        \vspace{2mm}
        \includegraphics[width=1cm]{arbeitsplatz} & Arbeitsplatz \\
        \includegraphics[width=1cm]{tisch} & Tisch \\
        \includegraphics[width=1cm]{durchgang} & Tür \\
        \includegraphics[width=1cm]{fenster} & Fenster \\
        \includegraphics[width=1cm]{trennwand} & Trennwand \\
    \end{tabular}
    \quad
    \begin{tabular}{ m{1.1cm} m{5cm} }
        \vspace{2mm}
        \includegraphics[width=1cm]{flächendesinfektion} & Flächendesinfektionsmittel \\
        \includegraphics[width=1cm]{handdesinfektion} & Händedesinfektionsmittel \\
        \includegraphics[width=1cm]{waschbecken} & Waschbecken \\
        \includegraphics[width=1cm]{hinweisschilder} & Hinweisschilder \\
    \end{tabular}
    }
\end{center}

\subsection{Future Interaction Lab}\label{app:plan_fil}

\frame{\includegraphics[width=\textwidth]{{fil_konzept}}}

\subsection{Eyetracking-Classroom}\label{app:plan_eyetracking}

\frame{\includegraphics[width=\textwidth]{{eyetracking-classroom_konzept}}}

\subsection{TechBase: VR4}\label{app:plan_vr4}

\frame{\includegraphics[width=\textwidth]{{vr4_konzept}}}


\end{document}
