\section{Allgemeine Informationen zu diesem Dokument}\label{sec:infos}

{
\singlespacing

Zu den Laboren des Lehrstuhls für Medieninformatik zählen:

\begin{itemize}
    \item Das Future Interaction Lab (Raumplan: Anhang \ref{app:plan_fil})
    \begin{itemize}
        \item Usability-Labor (PT 3.0.27)
        \item Werkstatt (PT 3.0.28)
        \item Besprechungsraum (PT 3.0.28A)
    \end{itemize}
    \item Der Eyetracking-Classroom (SG 5.20/21, Raumplan: Anhang \ref{app:plan_eyetracking})
    \begin{itemize}
        \item Eyetracking-Classroom i. e. S:
        \item Eyetracking-Server- und Arbeitsraum
    \end{itemize}
    \item Versuchsraum 4 in der TechBase (VR4, Raumplan: Anhang \ref{app:plan_vr4})
    \begin{itemize}
        \item VR4 Labor
        \item VR4 Studio
        \item VR4 Werkstatt
    \end{itemize}
\end{itemize}

\bigskip
\noindent
Dieses Dokument enthält

\begin{itemize}
    \item allgemeine Rahmenbedingungen und Hygieneregeln zur Nutzung der Labore,
    \item Maßnahmen des Lehrstuhls, um die Umsetzung der Hygienemaßnahmen in den Laboren zu gewährleisten,
    \item spezifische Hinweise zum Benutzen von Head-Mounted Displays (HMD),
    \item zusätzliche Regeln für die Durchführung von Nutzerstudien und
    \item individuelle Regelungen für jedes der Labore.
\end{itemize}

\bigskip
\noindent
Zusätzlich hängen diesem Dokument an:

\begin{itemize}
    \item Eine Liste mit Verantwortlichen für jedes Labor (Anhang \ref{app:verantwortliche})
    \item Hinweisschilder zu den Hygienemaßnahmen (Anhang \ref{app:hinweisschilder})
    \item Raumpläne der einzelnen Labore (Anhang \ref{app:raumpläne})
\end{itemize}

\bigskip
\noindent
Die in diesem Dokument aufgeführten Regeln und Maßnahmen gelten zusätzlich zu den bestehenden Sicherheitsvorschriften und Verhaltensregeln der Labore.
}
