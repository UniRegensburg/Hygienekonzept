\section{Durchführung von Nutzerstudien}

\noindent
In der Medieninformatik werden neue Interaktionstechniken und User Interfaces entwickelt und evaluiert, sowie bestehende untersucht und verbessert.
Ein etabliertes Vorgehen ist dabei das Durchführen von Nutzerstudien, bei denen Proband*Innen in einer kontrollierten Laborumgebung mit diesen Systemen interagieren.
Einige Studien werden auch in der virtuellen Realität durchgeführt und setzen spezielle Hardware wie Head-Mounted Displays oder Tracking-Anzüge für ein Motion-Capturing-System voraus.

\medskip
\noindent
Das Durchführen von Nutzerstudien im Labor ist nur dann gestattet, wenn die Laborinfrastruktur unbedingt benötigt wird und die Studie nicht remote durchgeführt werden kann.
Vorerst sind Interviews, Fokusgruppen und ähnliche Studien beispielsweise per Videochat durchzuführen.

\noindent
Beim Durchführen von Nutzerstudien sind folgende Maßnahmen zu beachten:

\begin{enumerate}
    \item Name, E-Mail-Adresse und Handynummer aller Proband*Innen, sowie der Zeitraum der Studie werden dokumentiert um mögliche Infektionsketten nachvollziehen zu können. Diese Daten werden nach 30 Tagen gelöscht.
    \item Nach jeder Proband*In werden alle Kontaktgegenstände und -flächen desinfiziert und, wenn möglich, der Laborraum für mindestens fünf Minuten gelüftet.
    \item Auch für Proband*Innen gilt: Befindet sich mehr als eine Person im Raum, muss ein Mund- und Nasenschutz getragen werden.
    \item Es wird jederzeit ein Mindestabstand von 1,5 Metern zu Probanden eingehalten.
    \item Nur ein/e Proband*In befindet sich gleichzeitig im Labor. Wenn möglich sollen Proband*Innen vor der Studie außerhalb des Gebäudes warten. Um Warteschlangen zu vermeiden, soll ausreichend viel Zeit zwischen einzelnen Proband*Innen eingeplant werden.
\end{enumerate}

\noindent
Nutzerstudien, die das Tragen von Hardware am Körper voraussetzen, weisen ein erhöhtes Risiko der Schmierinfektion auf.
Deshalb werden vorerst keine neuen Abschluss- und Projektarbeiten ausgeschrieben, bei denen diese Art von Studien durchgeführt werden.
Im Zuge laufender Arbeiten, deren Themen sich nicht mehr ohne Weiteres ändern lassen, dürfen unter erhöhten Sicherheitsmaßnahmen dennoch Nutzerstudien mit am Körper getragener Hardware durchgeführt werden.

\subsection{Durchführung von Studien mit Head-Mounted Displays}

Da durch die Verwendung von Head-Mounted Displays (HMD) ein höheres Infektionsrisiko besteht, ist bei Nutzerstudien mit derartigen Geräten ein genaues Ablaufprotokoll einzuhalten.

\subsubsection{Vorbereiten und Anbringen des HMD}

TODO

\subsubsection{Abnehmen des HMD}

TODO

\subsection{Durchführung von Studien mit Tracking-Anzügen}

Wird bei einer Studie das Motion-Capturing-System verwendet, so müssen Proband*Innen Ganzkörperanzüge mit speziellen Markern tragen.
Durch den direkten Körperkontakt besteht ein erhöhtes Infektionsrisiko, weshalb ein genaues Ablaufprotokoll zum An- und Ablegen des Anzugs, sowie dessen Reinigung und Aufbewahrung, einzuhalten ist.
