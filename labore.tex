\section{Individuelle Richtlinien für jedes der Labore}

\subsection{Future Interaction Lab: Usability-Labor}

\labinfo{Klassische Benutzerstudie, Entwicklungsarbeiten, Medienproduktion}{1}{Alexander Bazo, Christoph Härtl}

\noindent
Das Usability-Labor dient vorrangig der Durchführung von Nutzerstudien, insbesondere solcher Experimente, die eine “Alltagssituation” als Testumgebung erfordern.
Dazu sind neben einem klassischen Arbeitsplatz auch ein Sofa und Fernseher vorhanden.
Der Arbeitsplatz kann zusätzlich auch für Entwicklungsarbeiten und die Medienproduktion eingesetzt werden.
Bei Verwendung der vorhandenen Geräte (Workstations) muss nach Gebrauch eine Desinfektion der Eingabegeräte (Maus \& Tastatur) erfolgen.

\medskip
\noindent
Bei Verwendung des Raums für die Durchführung von Nutzerstudien wird die Fläche  durch Öffnen der Trennwand zu der angrenzenden Werkstatt erweitert.
Damit ist dann auch eine Trennung von Ein- und Ausgängen für die Testpersonen möglich.
Die räumlichen und einrichtungstechnischen  Beschränkungen im Labor schließen Studien mit mehr als einem Probanden aus.
Insgesamt sollten im Raum nicht mehr als zwei Personen (ProbandIn und TestleiterIn) anwesend sein.
Denkbar sind zusätzliche BeobachterIn in einem der angrenzenden Räume.
Entsprechende Videotechnik steht im Labor zur Verfügung.

\subsection{Future Interaction Lab: Werkstatt}

\labinfo{Entwicklungsarbeiten, Medienproduktion}{2}{Alexander Bazo, Christoph Härtl}

\noindent
Die Werkstatt im Future Interaction Lab dient vorrangig der (Software-) Entwicklungsarbeit.
Zusätzlich befindet sich hier eine ca. 2,5 x 2,5m große Freifläche, die für Tests- und Studien von VR-Arbeiten verwendet werden kann.
Die Arbeitsplätze sind an einer der Längsseiten angebracht.
Unter Berücksichtigung der notwendigen Abstandsregeln können hier bis zu zwei Personen gleichzeitig arbeiten.
Die Arbeitsflächen wurden weitestgehend freigeräumt, um eine einfache Reinigung und Desinfektion zu erlauben.
NutzerInnen werden angehalten, diesen Zustand beizubehalten.
Bei Verwendung der vorhandenen Geräte (Workstations) muss nach Gebrauch eine Desinfektion der Eingabegeräte (Maus \& Tastatur) erfolgen.
Hier kann zusätzliche Hardware für den regelmäßigen Austausch der angeschlossenen Geräte verwendet werden.
Grundsätzlich kann auch eine Nutzung der Workstations nur bei Verwendung eigener Eingabegeräte gestattet werden.

\medskip
\noindent
Bei Verwendung des Raums für die Durchführung von Nutzerstudien wird die Fläche  durch Öffnen der Trennwand zu einem der angrenzenden Räume (Besprechungsraum oder Usability-Raum) erweitert.
Damit ist dann auch eine Trennung von Ein- und Ausgängen für die Testpersonen möglich.

\subsection{Future Interaction Lab: Besprechungsraum}

\labinfo{Interviews, Projektbesprechung}{Konferenztisch mit 5 Sitzplätzen}{Alexander Bazo, Christoph Härtl}

TODO

\subsection{Eyetracking-Classroom}

\labinfo{Eye-Tracking-Experimente, auch mit gleichzeitiger Nutzung durch mehrere NutzerInnen}{4 (Zusätzliche Arbeitsplätze für Entwicklungsarbeit im Nebenraum vorhanden)}{Alexander Bazo (UR), Forian Hauser (OTH)}

\noindent
Im Classroom stehen 11 Hochleistungs-Eyetracker an separaten Arbeitsplätzen bereit.
Diese sind in Form eines klassischen CIP-Pool-Layouts angeordnet (Vier Sitzreihen mit je 2 bis 3 Arbeitsplätzen).
Bei entsprechender Einhaltung der Abstandsregel beim Eintritt in den Raum kann pro Sitzreihe ein Arbeitsplatz genutzt werden.
Im Idealfall können so auch Experimente durchgeführt werden, die eine gleichzeitige Nutzung des Labors durch mehrere Probanden erfordern.
Durch den angeschlossenen Nebenraum können separate Ein- und Ausgänge für den Laborraum geschaffen werden.
Nach der Verwendung der Arbeitsplätze werden diese desinfiziert, insbesondere die Tastaturen, Monitore und Oberflächen der Eye-Tracker.
Durch die Reduzierung der verwendeten Eye-Tracker pro Sitzreihe können die verwendeten Geräte zwischen einzelnen Studien alterniert werden.

\medskip

\noindent
\textbf{Jedwede Verwendung des Classrooms wird zwischen den Betreibern (OTH Regensburg, Prof. Mottok und Uni Regensburg, Prof. Wolff und Prof. Gruber) abgestimmt.}

\subsection{TechBase VR4: Labor}

\labinfo{Entwicklung von Prototypen, Arbeit mit interaktiven Tischen}{2}{Andreas Schmid, Raphael Wimmer}

\noindent
Das Labor ist der Hauptraum des VR4.
Er bietet genug Platz damit bis zu zwei Personen gleichzeitig darin arbeiten können, ohne den Mindestabstand zu unterschreiten.
In diesem Raum wird vorwiegend im Rahmen von Abschlussarbeiten an Projekten mit interaktiven Tischen und Projektionen gearbeitet, die auf die bestehende Laborinfrastruktur (beispielsweise Traversen) angewiesen sind.
Es werden persönliche Arbeitsbereiche für jede/n Labornutzer*In eingerichtet, welche nur von diesen benutzt werden.
Diese Arbeitsbereiche werden so platziert, dass noch genügend Abstand zum Durchgang Richtung Studio und Werkstatt besteht.

\subsection{TechBase VR4: Studio}

\labinfo{Entwicklung von VR-Anwendungen, Motion Capturing, Nutzerstudien}{2}{Martin Kocur, Polina Ugnivenko, Sarah Graf}
\subsection{TechBase VR4: Werkstatt}

\labinfo{Bau von Prototypen, Elektronikarbeitsplätze}{1}{Andreas Schmid, Raphael Wimmer}

\noindent
Die Werkstatt wird genutzt um Prototypen für neuartige Geräte zu bauen.
Dazu stehen Werkzeuge wie Lötstationen, Netzteile, ein Oszilloskop, Sägen und Akkuschrauber zur Verfügung.
Grundsätzlich ist die Nutzung der Werkstatt gestattet, jedoch kann aufgrund der geringen Größe nur eine Person gleichzeitig darin arbeiten.
Da Werkzeug und Bauteile teilweise schwer zu desinfizieren sind, wird auf große zeitliche Abstände bei der konsekutiven Nutzung geachtet um das Risiko von Schmierinfektionen zu verringern.
Die Tür vom Labor zur Werkstatt darf nur zum Betreten und Verlassen des Raums geöffnet werden.
